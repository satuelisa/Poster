\RequirePackage{xcolor}
\documentclass[a3]{sciposter}
\usepackage{graphicx}
\usepackage[spanish]{babel}   
\usepackage[utf8]{inputenc}

\title{Cómo se llama tu proyecto}
\author{Alumno de Verano y Elisa Schaeffer}
\institute {Posgrado en Ingeniería de Sistemas}
\email{alumno.verano@instituto.edu.mx}

\leftlogo[1]{uanl.png} 
\rightlogo[1]{fime.png}

\begin{document}

\conference{Verano Científico 2021 ---
  Facultad de Ingeniería Mecánica y Eléctrica ---
  Universidad Autónoma de Nuevo León}

\maketitle

\section{Introducción}

De qué se trata el proyecto. Hipótesis y objetivos. Motivación,
justificación.

\section{Antecedentes}

Conceptos y notación indispensables para que tus lectores puedan
entender el resto del trabajo.

\section{Estado de arte}

Qué han hecho los demás sobre este tema (citar a publicaciones
científicas, de preferencia publicadas en revistas que tengan un DOI y
que por lo menos algunos sean de los últimos cinco años). Si son
libros, que tengan un ISBN. Evitar citar puros sitios web.

Área de oportunidad: qué exactamente este trabajo contribuirá encima
de lo que ya existe. {\textquestiondown}Qué tiene de
diferente/original/impacto?

\section{Solución propuesta}

Metodología, herramientas (qué en sí haces, cómo lo haces, con qué lo
haces).

\section{Experimentos}

Diseño, reportaje y análisis de los resultados de los experimentos.

\section{Conclusiones}

Qué se logró hacer; qué posibilidad de trabajo a futuro se tiene para
este trabajo.

\end{document}
